%Terminal Command = xelatex -pdf FileName
%This is model of LaTeX for Chinese and Japanese
%KazukiAmakawa
%Pretreatment========================================================================================
\documentclass[UTF8]{ctexart}
\usepackage{graphicx}
\usepackage{multicol}
\usepackage{geometry}
\usepackage{lingmacros}
\usepackage{tree-dvips}
\usepackage{hyperref}
\usepackage{amsmath}
\usepackage{amssymb}
\usepackage{listings} 
\usepackage{amssymb}
\usepackage{verbatim}
\usepackage{cite}
\usepackage[amsmath,thmmarks]{ntheorem}
\usepackage{algpseudocode}
\usepackage{algorithm}
\usepackage{subfigure}
\usepackage{appendix}  

\theoremstyle{plain}
\theoremheaderfont{\normalfont\rmfamily\CJKfamily{hei}}
\theorembodyfont{\normalfont\rm\CJKfamily{kai}} \theoremindent0em
\theoremseparator{\hspace{1em}} \theoremnumbering{arabic}
\theoremsymbol{}
\newtheorem{theorem}{\textbf{定理}}[section]
\newtheorem{definition}{\textbf{定義}}[section]
\newtheorem{lemma}{\textbf{レンマ}}[section]

\geometry{left=2cm,right=2cm,top=2.4cm,bottom=2.2cm}
\title{タイトル}
\author{\textbf{天川かずき}}
\date{\today}

%Title================================================================================================
\begin{document}
\maketitle
\noindent \textbf{要旨}\\
\\

\noindent \textbf{キーワード} \\[20ex]
\thispagestyle{empty}
\newpage
\tableofcontents
\thispagestyle{empty}
\newpage
%Main=================================================================================================
%Section 1=================================================================================================
\setcounter{page}{1}
\newpage
\section{はじめに}
紹介の内容



%Reference============================================================================================
\newpage
\medskip
\bibliographystyle{plain}
\bibliography{/Users/kazukiamakawa/Desktop/お仕事関連/Paper/KazukiAmakawa.bib}

\noindent コード\texttt{$なし$}\\
\end{document}
%~~~~~~~~~~~~~~~~~~~~~~~~~~~~~~~~~~~~~~~~~~~~~~~~~~~~~~~~~~~~~~~~~~~~~~~~~~~~~~~~~~~~~~~~~~~~~~~~~~~~~



%Sample Codes
\begin{definition}
除环\\
对于一个集合$G$和两种二元运算:加法$+$和乘法 $\cdot$ ,如果$(G, +, \cdot)$为一个环,且\\
1) 对于任意一个非零元素,均存在它的逆元,即$\forall a \neq e \in G, \exists a^{-1} \in G\ s.t.\ a \cdot a^{-1} = 0$\\
则称$(G, +, \cdot)$为一个除环\\
\end{definition}


\begin{table}[!htb]
\centering  
\caption{传统代数与抽象代数的区别}  
\begin{tabular}{|c|c|c|c|c|}
\hline
传统代数 & 数   & $+, \cdot$ & 0,1   & 相反数,倒数\\
\hline
抽象代数 & 集合 & 二元运算     & 单位元 & 逆元 \\  
\hline
\end{tabular}  
\end{table}  


\noindent\rule[0.25\baselineskip]{\textwidth}{1pt}
\begin{lstlisting}[language = Matlab][basicstyle=\ttfamily][!htb]
[Code: Matlab]
clear
[number, txt, raw] = xlsread('File');
\end{lstlisting} 
\noindent\rule[0.25\baselineskip]{\textwidth}{1pt}


\begin{figure}[!htb]
\begin{center}
\includegraphics[width=1.00\textwidth]{Figure/MLResult.png} \\
图3 CARLA算法的迭代情况\\
\end{center}
\end{figure}


\begin{algorithm}[!htb]
\caption{CARLA algorithm main steps}
\begin{algorithmic}
1)
\end{algorithmic}
\end{algorithm}


\begin{flalign}
& & \nonumber\\
& & \nonumber
\end{flalign}
