%Pretreatment========================================================================================
\documentclass[12pt]{article}
\usepackage{lingmacros}
\usepackage{tree-dvips}
\usepackage{graphicx}
\usepackage{hyperref}
\usepackage{amsmath}
\usepackage{amssymb}
\usepackage{multicol}
\usepackage{geometry}
\usepackage{cite}
\usepackage[amsmath,thmmarks]{ntheorem}
\usepackage{algpseudocode}
\usepackage{algorithm}
\usepackage{listings} 
\usepackage{verbatim}
\usepackage{subfigure}
\usepackage{appendix}  

\theoremstyle{plain}
\theoremseparator{\hspace{1em}} \theoremnumbering{arabic}
\theoremsymbol{}
\newtheorem{theorem}{\textbf{Theorem}}[section]
\newtheorem{definition}{\textbf{Definition}}[section]
\newtheorem{lemma}{\textbf{Lemma}}[section]

\geometry{left=2cm,right=2cm,top=3cm,bottom=2cm}
\title{Title}
\author{Kazuki Amakawa}
\date{\today}

\begin{document}
\maketitle
\noindent \textbf{Abstract}\\

\noindent \textbf{Key Words:} \\
\newpage

%Main=================================================================================================
%Section 1============================================================================================
\section{Pretreatment}
We got 9 sample figures. For Figure 5, 6, 7, it is necessary for us to change the Contrast to make the figure clearly. And for Figure 9, it is too fuzzy, and we have to make it sharpen.
\subsection{Contrast Algorithm and Treatment}


%Section 2============================================================================================
\section{Image Binarization and Boundary Traversal}


%Reference============================================================================================
\newpage
\medskip
\bibliographystyle{plain}
\bibliography{/Users/kazukiamakawa/Desktop/お仕事関連/Paper/KazukiAmakawa.bib}

\end{document}
%~~~~~~~~~~~~~~~~~~~~~~~~~~~~~~~~~~~~~~~~~~~~~~~~~~~~~~~~~~~~~~~~~~~~~~~~~~~~~~~~~~~~~~~~~~~~~~~~~~~~~


%Sample Codes
\begin{definition}
Ring\\

\end{definition}


\begin{table}[!htb]
\centering  
\caption{Instruction}  
\begin{tabular}{|c|c|c|c|c|}
\hline
1 & 2   & $+, \cdot$ & 0,1   & 3\\
\hline
4 & 5 & 6     & 7 & 8 \\  
\hline
\end{tabular}  
\end{table}  


\noindent\rule[0.25\baselineskip]{\textwidth}{1pt}
\begin{lstlisting}[language = Matlab][basicstyle=\ttfamily][!htb]
[Code: Matlab]
clear
[number, txt, raw] = xlsread('File');
\end{lstlisting} 
\noindent\rule[0.25\baselineskip]{\textwidth}{1pt}


\begin{figure}[!htb]
\begin{center}
\includegraphics[width=1.00\textwidth]{Figure/MLResult.png} \\
图3 CARLA算法的迭代情况\\
\end{center}
\end{figure}


\begin{algorithm}[!htb]
\caption{CARLA algorithm main steps}
\begin{algorithmic}
1)
\end{algorithmic}
\end{algorithm}


\begin{flalign}
& & \nonumber\\
& & \nonumber
\end{flalign}